\documentclass[]{article}
\usepackage{lmodern}
\usepackage{amssymb,amsmath}
\usepackage{ifxetex,ifluatex}
\usepackage{fixltx2e} % provides \textsubscript
\ifnum 0\ifxetex 1\fi\ifluatex 1\fi=0 % if pdftex
  \usepackage[T1]{fontenc}
  \usepackage[utf8]{inputenc}
\else % if luatex or xelatex
  \ifxetex
    \usepackage{mathspec}
  \else
    \usepackage{fontspec}
  \fi
  \defaultfontfeatures{Ligatures=TeX,Scale=MatchLowercase}
\fi
% use upquote if available, for straight quotes in verbatim environments
\IfFileExists{upquote.sty}{\usepackage{upquote}}{}
% use microtype if available
\IfFileExists{microtype.sty}{%
\usepackage{microtype}
\UseMicrotypeSet[protrusion]{basicmath} % disable protrusion for tt fonts
}{}
\usepackage[margin=1in]{geometry}
\usepackage{hyperref}
\hypersetup{unicode=true,
            pdftitle={Assignment1},
            pdfauthor={Md Adnan, Ahmad Omar},
            pdfborder={0 0 0},
            breaklinks=true}
\urlstyle{same}  % don't use monospace font for urls
\usepackage{color}
\usepackage{fancyvrb}
\newcommand{\VerbBar}{|}
\newcommand{\VERB}{\Verb[commandchars=\\\{\}]}
\DefineVerbatimEnvironment{Highlighting}{Verbatim}{commandchars=\\\{\}}
% Add ',fontsize=\small' for more characters per line
\usepackage{framed}
\definecolor{shadecolor}{RGB}{248,248,248}
\newenvironment{Shaded}{\begin{snugshade}}{\end{snugshade}}
\newcommand{\AlertTok}[1]{\textcolor[rgb]{0.94,0.16,0.16}{#1}}
\newcommand{\AnnotationTok}[1]{\textcolor[rgb]{0.56,0.35,0.01}{\textbf{\textit{#1}}}}
\newcommand{\AttributeTok}[1]{\textcolor[rgb]{0.77,0.63,0.00}{#1}}
\newcommand{\BaseNTok}[1]{\textcolor[rgb]{0.00,0.00,0.81}{#1}}
\newcommand{\BuiltInTok}[1]{#1}
\newcommand{\CharTok}[1]{\textcolor[rgb]{0.31,0.60,0.02}{#1}}
\newcommand{\CommentTok}[1]{\textcolor[rgb]{0.56,0.35,0.01}{\textit{#1}}}
\newcommand{\CommentVarTok}[1]{\textcolor[rgb]{0.56,0.35,0.01}{\textbf{\textit{#1}}}}
\newcommand{\ConstantTok}[1]{\textcolor[rgb]{0.00,0.00,0.00}{#1}}
\newcommand{\ControlFlowTok}[1]{\textcolor[rgb]{0.13,0.29,0.53}{\textbf{#1}}}
\newcommand{\DataTypeTok}[1]{\textcolor[rgb]{0.13,0.29,0.53}{#1}}
\newcommand{\DecValTok}[1]{\textcolor[rgb]{0.00,0.00,0.81}{#1}}
\newcommand{\DocumentationTok}[1]{\textcolor[rgb]{0.56,0.35,0.01}{\textbf{\textit{#1}}}}
\newcommand{\ErrorTok}[1]{\textcolor[rgb]{0.64,0.00,0.00}{\textbf{#1}}}
\newcommand{\ExtensionTok}[1]{#1}
\newcommand{\FloatTok}[1]{\textcolor[rgb]{0.00,0.00,0.81}{#1}}
\newcommand{\FunctionTok}[1]{\textcolor[rgb]{0.00,0.00,0.00}{#1}}
\newcommand{\ImportTok}[1]{#1}
\newcommand{\InformationTok}[1]{\textcolor[rgb]{0.56,0.35,0.01}{\textbf{\textit{#1}}}}
\newcommand{\KeywordTok}[1]{\textcolor[rgb]{0.13,0.29,0.53}{\textbf{#1}}}
\newcommand{\NormalTok}[1]{#1}
\newcommand{\OperatorTok}[1]{\textcolor[rgb]{0.81,0.36,0.00}{\textbf{#1}}}
\newcommand{\OtherTok}[1]{\textcolor[rgb]{0.56,0.35,0.01}{#1}}
\newcommand{\PreprocessorTok}[1]{\textcolor[rgb]{0.56,0.35,0.01}{\textit{#1}}}
\newcommand{\RegionMarkerTok}[1]{#1}
\newcommand{\SpecialCharTok}[1]{\textcolor[rgb]{0.00,0.00,0.00}{#1}}
\newcommand{\SpecialStringTok}[1]{\textcolor[rgb]{0.31,0.60,0.02}{#1}}
\newcommand{\StringTok}[1]{\textcolor[rgb]{0.31,0.60,0.02}{#1}}
\newcommand{\VariableTok}[1]{\textcolor[rgb]{0.00,0.00,0.00}{#1}}
\newcommand{\VerbatimStringTok}[1]{\textcolor[rgb]{0.31,0.60,0.02}{#1}}
\newcommand{\WarningTok}[1]{\textcolor[rgb]{0.56,0.35,0.01}{\textbf{\textit{#1}}}}
\usepackage{graphicx,grffile}
\makeatletter
\def\maxwidth{\ifdim\Gin@nat@width>\linewidth\linewidth\else\Gin@nat@width\fi}
\def\maxheight{\ifdim\Gin@nat@height>\textheight\textheight\else\Gin@nat@height\fi}
\makeatother
% Scale images if necessary, so that they will not overflow the page
% margins by default, and it is still possible to overwrite the defaults
% using explicit options in \includegraphics[width, height, ...]{}
\setkeys{Gin}{width=\maxwidth,height=\maxheight,keepaspectratio}
\IfFileExists{parskip.sty}{%
\usepackage{parskip}
}{% else
\setlength{\parindent}{0pt}
\setlength{\parskip}{6pt plus 2pt minus 1pt}
}
\setlength{\emergencystretch}{3em}  % prevent overfull lines
\providecommand{\tightlist}{%
  \setlength{\itemsep}{0pt}\setlength{\parskip}{0pt}}
\setcounter{secnumdepth}{0}
% Redefines (sub)paragraphs to behave more like sections
\ifx\paragraph\undefined\else
\let\oldparagraph\paragraph
\renewcommand{\paragraph}[1]{\oldparagraph{#1}\mbox{}}
\fi
\ifx\subparagraph\undefined\else
\let\oldsubparagraph\subparagraph
\renewcommand{\subparagraph}[1]{\oldsubparagraph{#1}\mbox{}}
\fi

%%% Use protect on footnotes to avoid problems with footnotes in titles
\let\rmarkdownfootnote\footnote%
\def\footnote{\protect\rmarkdownfootnote}

%%% Change title format to be more compact
\usepackage{titling}

% Create subtitle command for use in maketitle
\providecommand{\subtitle}[1]{
  \posttitle{
    \begin{center}\large#1\end{center}
    }
}

\setlength{\droptitle}{-2em}

  \title{Assignment1}
    \pretitle{\vspace{\droptitle}\centering\huge}
  \posttitle{\par}
    \author{Md Adnan, Ahmad Omar}
    \preauthor{\centering\large\emph}
  \postauthor{\par}
      \predate{\centering\large\emph}
  \postdate{\par}
    \date{5/18/2020}


\begin{document}
\maketitle

Exercise 1.1

\begin{Shaded}
\begin{Highlighting}[]
\KeywordTok{library}\NormalTok{(}\StringTok{"ggplot2"}\NormalTok{)}
\KeywordTok{library}\NormalTok{(}\StringTok{"dplyr"}\NormalTok{)}
\end{Highlighting}
\end{Shaded}

\begin{verbatim}
## 
## Attaching package: 'dplyr'
\end{verbatim}

\begin{verbatim}
## The following objects are masked from 'package:stats':
## 
##     filter, lag
\end{verbatim}

\begin{verbatim}
## The following objects are masked from 'package:base':
## 
##     intersect, setdiff, setequal, union
\end{verbatim}

\begin{Shaded}
\begin{Highlighting}[]
\KeywordTok{library}\NormalTok{(}\StringTok{"factoextra"}\NormalTok{)}
\end{Highlighting}
\end{Shaded}

\begin{verbatim}
## Warning: package 'factoextra' was built under R version 3.6.3
\end{verbatim}

\begin{verbatim}
## Welcome! Want to learn more? See two factoextra-related books at https://goo.gl/ve3WBa
\end{verbatim}

\begin{Shaded}
\begin{Highlighting}[]
\NormalTok{pca_toy <-}\StringTok{ }\KeywordTok{read.delim}\NormalTok{(}\StringTok{"A://AI/Process/pca_toy.txt"}\NormalTok{)}
\end{Highlighting}
\end{Shaded}

\begin{enumerate}
\def\labelenumi{(\alph{enumi})}
\tightlist
\item
  It is necesssary to standardize data before principle Component
  Analysis because standardization of all the variables of a dataset
  will have same standard deviation. In this analysis, we found that PC1
  has a standard deviation of 1.72. Thus, for PC1, all variables will
  have same weight and will be plotted against a single axis for.
\end{enumerate}

\begin{Shaded}
\begin{Highlighting}[]
\CommentTok{#Standardization of pca_toy dataset}
\NormalTok{pca_df <-}\StringTok{ }\KeywordTok{data.frame}\NormalTok{(}\KeywordTok{scale}\NormalTok{(pca_toy))}
\NormalTok{myPr <-}\StringTok{ }\KeywordTok{prcomp}\NormalTok{(pca_df)}

\KeywordTok{head}\NormalTok{(pca_df)}
\end{Highlighting}
\end{Shaded}

\begin{verbatim}
##            a           b          c          d
## 1  2.0613318  1.37030688  0.9363476  0.3742048
## 2  0.8989266  1.37030688  0.5107350  0.8798869
## 3  1.8675976  1.05163086  1.3619601  0.8798869
## 4 -0.8446811 -1.49777728 -0.5532963 -0.1314774
## 5  1.0926608  0.09560281  0.7235413  0.8798869
## 6 -0.4572127  0.09560281  0.5107350 -0.1314774
\end{verbatim}

\begin{enumerate}
\def\labelenumi{(\alph{enumi})}
\setcounter{enumi}{1}
\item
\end{enumerate}

\begin{Shaded}
\begin{Highlighting}[]
\CommentTok{#scatter plot with PC1 and PC2}




\NormalTok{pca <-}\StringTok{ }\KeywordTok{cbind}\NormalTok{(pca_df, myPr}\OperatorTok{$}\NormalTok{x[,}\DecValTok{1}\OperatorTok{:}\DecValTok{2}\NormalTok{])}



\KeywordTok{ggplot}\NormalTok{(pca, }\KeywordTok{aes}\NormalTok{(PC1, PC2)) }\OperatorTok{+}
\StringTok{  }\KeywordTok{stat_ellipse}\NormalTok{(}\DataTypeTok{geom =} \StringTok{"polygon"}\NormalTok{, }\DataTypeTok{col =}\StringTok{"black"}\NormalTok{, }\DataTypeTok{alpha =} \FloatTok{0.5}\NormalTok{)}\OperatorTok{+}
\StringTok{  }\KeywordTok{geom_point}\NormalTok{(}\DataTypeTok{shape =} \DecValTok{21}\NormalTok{, }\DataTypeTok{col =} \StringTok{"black"}\NormalTok{)}
\end{Highlighting}
\end{Shaded}

\includegraphics{PCA_markdown_files/figure-latex/unnamed-chunk-3-1.pdf}

\begin{enumerate}
\def\labelenumi{(\alph{enumi})}
\setcounter{enumi}{2}
\tightlist
\item
  ``Barplot of PC1 Percentage'' shows that variable ``c'' contributes
  for above 25\% of PC1. ``Barplot of PC2 Percentage'' shows that
  variable ``b'' contributes for above 40\% of total variables in PC2.
  It can be stated that ``c'' and ``b'' highly contributes to PC1 and
  PC2, respectively.
\end{enumerate}

\begin{Shaded}
\begin{Highlighting}[]
\CommentTok{#determining responsible variables}
\NormalTok{var <-}\StringTok{ }\KeywordTok{get_pca_var}\NormalTok{(myPr)}

\KeywordTok{barplot}\NormalTok{(var}\OperatorTok{$}\NormalTok{contrib[,}\DecValTok{1}\NormalTok{], }\DataTypeTok{main =} \StringTok{"Barplot of PC1 Percentage"}\NormalTok{, }\DataTypeTok{xlab =} \StringTok{"Percent"}\NormalTok{, }\DataTypeTok{ylab =} \StringTok{"Variables"}\NormalTok{)}
\end{Highlighting}
\end{Shaded}

\includegraphics{PCA_markdown_files/figure-latex/unnamed-chunk-4-1.pdf}

\begin{Shaded}
\begin{Highlighting}[]
\KeywordTok{barplot}\NormalTok{(var}\OperatorTok{$}\NormalTok{contrib[,}\DecValTok{2}\NormalTok{], }\DataTypeTok{main =} \StringTok{"Barplot of PC2 Percentage"}\NormalTok{, }\DataTypeTok{xlab =} \StringTok{"Percent"}\NormalTok{, }\DataTypeTok{ylab =} \StringTok{"Variables"}\NormalTok{)}
\end{Highlighting}
\end{Shaded}

\includegraphics{PCA_markdown_files/figure-latex/unnamed-chunk-4-2.pdf}

\begin{enumerate}
\def\labelenumi{(\alph{enumi})}
\setcounter{enumi}{3}
\tightlist
\item
  A total of 86.9\% variation has been explained by PC1 and PC2. The
  importance of components chart below shows that PC1 correspond to
  0.7316 of the variation, followed by PC2 with 0.1366. ``My PCA
  Graph'', shows PC1 corresponds to 73.2\% and PC2 corresponds to 13.7\%
  of variations in this dataset.
\end{enumerate}

\begin{Shaded}
\begin{Highlighting}[]
\CommentTok{#to determine the percentage of variance }
\NormalTok{myPr.var <-}\StringTok{ }\NormalTok{myPr}\OperatorTok{$}\NormalTok{sdev}\OperatorTok{^}\DecValTok{2}
\NormalTok{myPr.var.par <-}\StringTok{ }\KeywordTok{round}\NormalTok{(myPr.var}\OperatorTok{/}\KeywordTok{sum}\NormalTok{(myPr.var)}\OperatorTok{*}\DecValTok{100}\NormalTok{,}\DecValTok{1}\NormalTok{)}


\KeywordTok{barplot}\NormalTok{(myPr.var.par, }\DataTypeTok{main =} \StringTok{"Scree Plot"}\NormalTok{, }\DataTypeTok{xlab =} \StringTok{"Principal Component"}\NormalTok{,}
        \DataTypeTok{ylab =} \StringTok{"Percent Variation"}\NormalTok{)}
\end{Highlighting}
\end{Shaded}

\includegraphics{PCA_markdown_files/figure-latex/unnamed-chunk-5-1.pdf}

\begin{Shaded}
\begin{Highlighting}[]
\NormalTok{p=}\KeywordTok{data.frame}\NormalTok{(myPr}\OperatorTok{$}\NormalTok{x)}

\NormalTok{myPr.data <-}\StringTok{ }\KeywordTok{data.frame}\NormalTok{(}\DataTypeTok{sample=}\KeywordTok{rownames}\NormalTok{(p),}
                        \DataTypeTok{X=}\NormalTok{p[,}\DecValTok{1}\NormalTok{],}
                        \DataTypeTok{Y=}\NormalTok{p[,}\DecValTok{2}\NormalTok{])}
\KeywordTok{ggplot}\NormalTok{(}\DataTypeTok{data=}\NormalTok{myPr.data, }\KeywordTok{aes}\NormalTok{(}\DataTypeTok{x=}\NormalTok{X, }\DataTypeTok{y=}\NormalTok{Y, }\DataTypeTok{label=}\NormalTok{sample))}\OperatorTok{+}
\StringTok{  }\KeywordTok{geom_text}\NormalTok{()}\OperatorTok{+}
\StringTok{  }\KeywordTok{xlab}\NormalTok{(}\KeywordTok{paste}\NormalTok{(}\StringTok{"PC1 - "}\NormalTok{, myPr.var.par[}\DecValTok{1}\NormalTok{], }\StringTok{"%"}\NormalTok{, }\DataTypeTok{sep=}\StringTok{""}\NormalTok{))}\OperatorTok{+}
\StringTok{  }\KeywordTok{ylab}\NormalTok{(}\KeywordTok{paste}\NormalTok{(}\StringTok{"PC2 - "}\NormalTok{, myPr.var.par[}\DecValTok{2}\NormalTok{], }\StringTok{"%"}\NormalTok{, }\DataTypeTok{sep=}\StringTok{""}\NormalTok{))}\OperatorTok{+}
\StringTok{  }\KeywordTok{theme_bw}\NormalTok{()}\OperatorTok{+}
\StringTok{  }\KeywordTok{ggtitle}\NormalTok{(}\StringTok{"My PCA Graph"}\NormalTok{)}
\end{Highlighting}
\end{Shaded}

\includegraphics{PCA_markdown_files/figure-latex/unnamed-chunk-5-2.pdf}

\begin{Shaded}
\begin{Highlighting}[]
\KeywordTok{summary}\NormalTok{(myPr)}
\end{Highlighting}
\end{Shaded}

\begin{verbatim}
## Importance of components:
##                           PC1    PC2     PC3    PC4
## Standard deviation     1.7107 0.7391 0.62849 0.3639
## Proportion of Variance 0.7316 0.1366 0.09875 0.0331
## Cumulative Proportion  0.7316 0.8681 0.96690 1.0000
\end{verbatim}


\end{document}
